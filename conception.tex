\documentclass{article}
\usepackage{multimedia}
\usepackage{polski}
\usepackage[utf8]{inputenc}
\usepackage[T1]{fontenc}
\usepackage{color}
\usepackage{amsthm}
\usepackage{graphicx}
\usepackage{float}
\usepackage{afterpage}

\renewcommand*{\figurename}{\textit{Rysunek}}
\renewcommand{\thefigure}{\textit{\arabic{figure}}}
\renewcommand*{\tablename}{\textit{Tabela}}
\renewcommand{\thetable}{\textit{\arabic{figure}}}


\begin{document}

\begin{titlepage}
	%TODO: jak widać
	\vspace{2em}{\centering\large{Żadna pracownia xD}\par}

	\vspace{2em}{\centering\large{Instytut Informatyki Uniwersytetu Wrocławskiego}\par}

  \vspace{17em}{\centering\large{Agnieszka Dudek, Piotr Kowalczyk}\par}

	\vspace{2em}{\centering\huge{Dokumentacja projektu Tablica efektów kształcenia}\par}

	\vspace{2em}{\centering\Large{Koncepcja wykonania systemu}\par}

	\vspace{17em}{\centering\large{Wrocław, \today}\par}

	\vspace{1em}{\centering\large{Wersja 0.1}\par}

\end{titlepage}

\addtocounter{page}{1}
\newpage

\begin{table}[h!]
	\begin{center}
		\caption{Historia zmian dokonywanych w dokumencie}
		\begin{tabular}{|l|l|l|l|}
			\hline
			Data & Numer wersji & Opis & Autor \\
		  \hline \hline
      2018-11-26 & 0.1 & Utworzenie dokumentu & Piotr Kowalczyk \\
      \hline
    \end{tabular}
	\end{center}
\end{table}	

\tableofcontents

\newpage


\section{Wprowadzenie}

\subsection{Cel dokumentu}
Niniejszy dokument ma na celu precyzyjny opis koncepcji wykonania projektu Tablica efektów kształcenia. Stworzony jest na potrzeby przedmiotu inżynieria oprogramowania.



\section{Scenariusze przypadków użycia} %TODO: jakoś mądrzej...?
Odpowiadają one historyjkom użytkownika opisanym w dokumencie Specyfikacja wymagań.

\subsection{Pierwszy scenariusz}
Po zalogowaniu do Systemu Zapisów widoczny jest przycisk podpisany Tablica efektów (dalej zwany po prostu przyciskiem). Znajduje się on obok przycisków Przedmioty, Plan zajęć itp. Po kliknięciu w przycisk pojawia się ekran, na którym znajdują się:
\begin{itemize}
	\item lista efektów kształcenia czekających na zrealizowanie
	\item lista zrealizowanych efektów kształcenia
\end{itemize}

\subsection{Drugi scenariusz}
%TODO: poprawić (jw.), gdy zostanie napisana druga historyjka

\subsection{Trzeci scenariusz}
Zmiana przedmiotu w USOSie powoduje, że w następnej migracji danych aktualizowane są dane w Systemie Zapisów dotyczące efektów kształcenia.


%TODO: skontrolować strony
\afterpage{\null\newpage}
\newpage

\section{Projekty ekranów dla przypadków użycia}
Odpowiadają one scenariuszom opisanym w powyżej.

\subsection{Pierwszy scenariusz}
\begin{figure}[H]
	\begin{center}
		\caption{Projekt przycisku Efekty kształcenia}
		\includegraphics[scale=0.5]{te.png}
	\end{center}
\end{figure}

%TODO: grafika tabeli na stronie, najlepiej robiona w paincie

\subsection{Drugi scenariusz}
%TODO: poprawić (jw.), gdy zostanie napisana druga historyjka

\subsection{Trzeci scenariusz}
Nie wymaga projektów międzymordzia, %TODO: usunąć ten żart
ponieważ całość już istnieje.



\section{Punkt 3. w zadaniu}
Nie wiem o co tu chodzi. %TODO: zrobić

\subsection{Model konceptualny rzeczywistości, której dotyczy aplikacja, tzn. identyfikacja encji i powiązań oraz warunków dotyczących powiazań}
??? %TODO: zrobić

\subsection{Wymienienie oraz przedstawienie graficzne podstawowych elementów aplikacji oraz
powiązań między nimi}
% - sprzęt,
% - oprogramowanie systemowe, bazy danych, narzędzia
% programistyczne, oprogramowanie do automatycznego testowania.
% - struktury logicznej oprogramowania obiektowego
% ( podział kodu na główne komponenty np. klasy i powiązania
% miedzy nimi)}
%TODO: use paint

\subsection{Omówienie interfejsów aplikacji z otoczeniem}
Głównym interfejsem aplikacji jest strona Systemu Zapisów. Oprócz tego ważne jest powiązanie z USOSem. Jest także furtka, służąca do ręcznych poprawek w sytuacjach wyjątkowych (takich, które nie są skorelowane ze zmianami w USOSie).



%TODO: skontrolować strony
\afterpage{\null\newpage}
\newpage

\section{Schemat bazy danych}
%TODO: wybrać
%v1
Baza danych składa się z jednej tablicy, w której kluczami są identyfikatory studentów, a efekty kształcenia kolejnymi polami typu boolean.
%v2
Baza danych składa się z tylu tablic, ile jest toków studiów na kierunku informatyka na naszej uczelni. Kluczami są identyfikatory studentów, a efekty kształcenia kolejnymi polami typu boolean (nie da się być jednocześnie na dwóch tokach studiów).

Baza jest tworzona (za pierwszym razem) i aktualizowana podczas migracji danych z USOSa.



\section{Zasady kodowania}
%TODO: jak jest ok, to usuń ten komentarz
Podczas kodowania będą obowiązywać zasady z Systemu Zapisów, a w przypadki tamże niezdefiniowane rozstrzygane będą zgodnie z normą wypracowaną przez Google\footnote{https://google.github.io/styleguide/jsguide.html}.



\section{Ryzyko}

\subsection{Identyfikacja ryzyka} %TODO: przepisać ładniej
Nie musimy się bać, że nasze oprogramowanie przestanie być potrzebne. Nie musimy się bać, że stracimy odbiorców. Ale za to wykonanie projektu może się opóźnić i tego się głównie boimy.

\subsection{Zasady zarządzania ryzykiem}
%TODO: też nie wiem o co tu chodzi



\section{Ocena zgodności wykonanych prac z planami}
Nie ma co zepsuć, bo jak coś zmienimy to już będzie zupełnie inny projekt. Wygląd se można zmienić i tyle.
%TODO: poprawić


\end{document}
