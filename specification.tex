\documentclass{article}
\usepackage{multimedia}
\usepackage{polski}
\usepackage[utf8]{inputenc}
\usepackage[T1]{fontenc}
\usepackage{color}
\usepackage{amsthm}
\usepackage[labelsep=period]{caption}

\renewcommand*{\tablename}{Tabela}
\renewcommand{\thetable}{\arabic{figure}}
\renewcommand*{\thesection}{\arabic{section}.}
\renewcommand*{\thesubsection}{\arabic{section}.\arabic{subsection}.}


\begin{document}

\begin{titlepage}
	\vspace{2em}{\centering\large{Instytut Informatyki Uniwersytetu Wrocławskiego}\par}

  \vspace{17em}{\centering\large{Agnieszka Dudek, Piotr Kowalczyk}\par}

	\vspace{2em}{\centering\huge{Dokumentacja projektu \textit{Tablica efektów kształcenia}}\par}

	\vspace{4em}{\centering\Large{Specyfikacja wymagań}\par}

	\vspace{17em}{\centering\large{Wrocław, \today}\par}

	\vspace{1em}{\centering\large{Wersja 0.5}\par}

\end{titlepage}

\addtocounter{page}{1}
\newpage

\begin{table}[h!]
	\begin{center}
		\caption{Historia zmian dokonywanych w dokumencie}
		\begin{tabular}{|l|l|l|l|}
			\hline
			Data & Numer wersji & Opis & Autor \\
			\hline \hline
			2018-11-26 & 0.1 & Utworzenie dokumentu & Piotr Kowalczyk \\
			\hline 
			2018-11-30 & 0.2 & Korekta dokumentu & Agnieszka Dudek \\
			\hline
			2018-12-05 & 0.3 & Korekta dokumentu & Piotr Kowalczyk \\
			\hline
			2018-12-13 & 0.4 & Korekta dokumentu & Agnieszka Dudek \\
			\hline
			2018-12-15 & 0.5 & Korekta dokumentu & Agnieszka Dudek \\
			\hline

    \end{tabular}
	\end{center}
\end{table}	

\tableofcontents

\newpage


\section{Wprowadzenie}

\subsection{Cel dokumentu}
Niniejszy dokument ma na celu precyzyjne określenie wymagań projektu \textit{Tablica efektów kształcenia}.
Dokument stanowi pierwsze z listopadowej grupy zadań realizowanych w ramach pracowni z inżynierii oprogramowania.


\section{Historyjki użytkownika}
\subsection{Pierwsza historyjka}
Jako student informatyki, który właśnie układa swój plan zajęć, chcę sprawdzić, ile punktów ECTS z danych grup przedmiotów już zdobyłem.
Dzięki temu w~kolejnym semestrze będę mógł wybrać takie przedmioty, które uzupełnią punkty brakujące do wymaganego limitu.
\subsection{Druga historyjka}
Jako studentka informatyki, która zaczyna ostatni semestr studiów, chcę zobaczyć, jakich efektów kształcenia jeszcze nie osiągnęłam i upewnić się, że wszystkie wymagania spełnię w ostatnim półroczu i na czas będę mogła skończyć studia.
\subsection{Trzecia historyjka}
Jako osoba zatrudniona w dziekanacie dowiaduję się, że student wypracował zaliczenie z przedmiotu \textit{Algorytmy i struktury danych} w lutym, prosi zatem o przedłużenie sesji oraz korektę danych. 
Chcę, aby po zmianie oceny oraz efektów kształcenia w USOS zauktualizowane zostały dane dotyczące efektów kształcenia w Systemie Zapisów, aby nie było konieczności ręcznego poprawiania danych.


\section{Wymagania funkcjonalne}

\subsection{Pokazanie \textit{Tablicy efektów} użytkownikowi Systemu Zapisów}
Jest to najważniejsza funkcja projektowanego systemu.
Po zleceniu systemowi wyświetlenia \textit{Tablicy efektów} użytkownik powienien niezwłocznie zobaczyć:
\begin{itemize}
	\item listę osiągniętych efektów kształcenia,
	\item listę nieosiągniętych efektów koniecznych do ukończenia aktualnego etapu studiów,
	\item punkty ECTS zdobyte z przedmiotów humanistycznych oraz informatycznych, a także z projektów i kursów.
\end{itemize}

\subsection{Generowanie danych do pliku}
Drobną, lecz często bardzo przydatną funkcją będzie możliwość utworzenia \textit{Tablicy efektów} w pliku w formacie PDF oraz jej wydrukowania.
Będzie to użyteczne zarówno dla studentów, którzy wolą analizować dane na papierze niż na komputerze, jak i dla pań z dziekanatu, do których jeszcze przez jakiś czas będą przychodzić studenci z prośbą o dostarczenie wiarygodnych i kompletnych informacji o osiągniętych efektach.

\subsection{Importowanie danych z USOS} %obsługi studiów
Ręczne wpisywanie danych wszystkich studentów korzystających z Systemu Zapisów byłoby zaprzeczeniem wszystkiego, w co wierzymy.
Projektowany system musi automatycznie synchronizować dane dotyczące efektów kształcenia studentów z USOS.

\subsection{Ręczna modyfikacja danych}
Zdarza się, że student osiąga któryś z efektów kształcenia, zaliczając przedmiot po terminie lub bez uczęszczania na jakikolwiek przedmiot (np. możliwe jest zwolnienie z projektu zespołowego na podstawie pracy zawodowej).
Zdarzają się także błędy nawet w najdoskonalszych systemach informatycznych, a USOS trudno byłoby zaliczyć do tej grupy.
Konieczna jest zatem możliwość ręcznej modyfikacji danych, które znajdują się w Systemie Zapisów i dotyczą efektów kształcenia.


\section{Wymagania niefunkcjonalne}

\subsection{Płynność działania}
Przez większość czasu z Systemu Zapisów korzysta niewielka liczba studentów, jednak zmienia się to po otwarciu zapisów na dany semestr oraz w trakcie głosowania.
W tych okresach \textit{Tablica efektów kształcenia} będzie szczególnie często użytkowana.

Aby sprawdzić, czy \textit{Tablica efektów kształcenia} na pewno będzie działała płynnie, należy ją przetestować, wysyłając do niej 600 zapytań w ciągu 1 minuty.
Nasz system powinien działać sprawnie w każdym przypadku, nawet wówczas gdyby wszyscy studenci chcieli z niego skorzystać w tym samym momencie (obecnie w naszej uczelni na kierunku informatyka studiuje 525 osób).
%uwaga - który system?

\subsection{Przejrzystość}
Wyświetlane dane mają studentom przede wszystkim pomóc.
Zależy nam na tym, aby były przedstawione w sposób przejrzysty i zrozumiały.
Aby upewnić się, że rzeczywiście tak będzie, należy skonsultować ze studentami i pracownikami dziekanatu proponowany format przedstawianych informacji.

\end{document}
