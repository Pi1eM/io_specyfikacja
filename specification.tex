\documentclass{article}
\usepackage{multimedia}
\usepackage{polski}
\usepackage[utf8]{inputenc}
\usepackage[T1]{fontenc}
\usepackage{color}
\usepackage{amsthm}

\renewcommand*{\tablename}{\textit{Tabela}}
\renewcommand{\thetable}{\textit{\arabic{figure}}}


\begin{document}

\begin{titlepage}
	%TODO: jak widać
	\vspace{2em}{\centering\large{Żadna pracownia xD}\par}

	\vspace{2em}{\centering\large{Instytut Informatyki Uniwersytetu Wrocławskiego}\par}

  \vspace{17em}{\centering\large{Agnieszka Dudek, Piotr Kowalczyk}\par}

	\vspace{2em}{\centering\huge{Dokumentacja projektu Tablica efektów kształcenia}\par}

	\vspace{2em}{\centering\Large{Specyfikacja wymagań}\par}

	\vspace{17em}{\centering\large{Wrocław, \today}\par}

	\vspace{1em}{\centering\large{Wersja 0.1}\par}

\end{titlepage}

\addtocounter{page}{1}
\newpage

\begin{table}[h!]
	\begin{center}
		\caption{Historia zmian dokonywanych w dokumencie}
		\begin{tabular}{|l|l|l|l|}
			\hline
			Data & Numer wersji & Opis & Autor \\
		  \hline \hline
      2018-11-26 & 0.1 & Utworzenie dokumentu & Piotr Kowalczyk \\
      \hline
    \end{tabular}
	\end{center}
\end{table}	

\tableofcontents

\newpage


\section{Wprowadzenie}

\subsection{Cel dokumentu}
Niniejszy dokument ma na celu precyzyjne określenie wymagań projektu Tablica efektów kształcenia. Stworzony jest na potrzeby przedmiotu inżynieria oprogramowania.



\section{Historyjki użytkownika}
%TODO: zrobić to porządnie, nie tak jak teraz

\subsection{Pierwsza historyjka}
Jako student chcę sprawdzić, które efekty kształcenia już osiągnąłem, aby móc zaszpanować przed moimi ziomkami.

\subsection{Druga historyjka} %TODO: napisać coś mądrego
To samo, tylko chcę pokazać rodzicom, że dobrze się uczę.

\subsection{Trzecia historyjka}
Jako osoba zatrudniona w dziekanacie dowiaduję się, że student wypracował zaliczenie z przedmiotu algorytmy i struktury danych w lutym, prosi zatem o przedłużenie sesji oraz korektę danych. Chcę, aby po zmianie oceny oraz efektów kształcenia w USOSie zauktualizowane zostały dane nt. efektów kształcenia w Systemie Zapisów, aby nie było konieczności ręcznego poprawiania danych.



\section{Wymagania funkcjonalne}

\subsection{Pokazanie tablicy efektów użytkownikowy Systemu Zapisów}
Jest to kluczowa funkcja projektowanego systemu. Po zleceniu systemowi wyświetlenia tablicy efektów użytkownik powienien niezwłocznie zobaczyć:
\begin{itemize}
	\item listę osiągniętych efektów kształcenia,
	\item listę nieosiągniętych efektów koniecznych do ukończenia aktualnego etapu studiów.
\end{itemize}


\subsection{Importowanie danych z USOSa}
Ręczne wpisywanie danych wszystkich studentów korzystających z Systemu Zapisów byłoby zaprzeczeniem wszystkiego, w co wierzymy. Projektowany system musi automatycznie synchronizować dane dotyczące efektów kształcenia studentów z USOSem.


\subsection{Ręczna modyfikacja danych}
Zdarza się, że student osiąga któryś z efektów kształcenia zaliczając przedmiot po terminie bądź bez uczęszczania na jakikolwiek przedmiot (np. możliwe jest zwolnienie z projektu zespołowego na podstawie pracy zawodowej). Zdarzają się także błędy nawet w najdoskonalszych systemach informatycznych, a USOSa trudno byłoby zaliczyć do tej grupy. Konieczna jest zatem możliwość ręcznej modyfikacji danych, które znajdują się w Systemie Zapisów i dotyczą efektów kształcenia.



\section{Wymagania niefunkcjonalne}

\subsection{Płynność działania} %TODO: jakaś wymierna miara?
Przez większość czasu z Systemu Zapisów korzysta niewielka ilość studentów, jednak zmienia się to po otwarciu zapisów na dany semestr oraz w trakcie głosowania. W tych okresach Tablica efektów kształcenia będzie szczególnie często użytkowana.



\end{document}
