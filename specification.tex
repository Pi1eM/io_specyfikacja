\documentclass{article}
\usepackage{multimedia}
\usepackage{polski}
\usepackage[utf8]{inputenc}
\usepackage[T1]{fontenc}
\usepackage{color}
\usepackage{amsthm}

\renewcommand*{\tablename}{\textit{Tabela}}
\renewcommand{\thetable}{\textit{\arabic{figure}}}


\begin{document}

\begin{titlepage}
	\vspace{2em}{\centering\large{Instytut Informatyki Uniwersytetu Wrocławskiego}\par}

  \vspace{17em}{\centering\large{Agnieszka Dudek, Piotr Kowalczyk}\par}

	\vspace{2em}{\centering\huge{Dokumentacja projektu \textit{Tablica efektów kształcenia}}\par}

	\vspace{4em}{\centering\Large{Specyfikacja wymagań}\par}

	\vspace{17em}{\centering\large{Wrocław, \today}\par}

	\vspace{1em}{\centering\large{Wersja 0.2}\par}

\end{titlepage}

\addtocounter{page}{1}
\newpage

\begin{table}[h!]
	\begin{center}
		\caption{Historia zmian dokonywanych w dokumencie}
		\begin{tabular}{|l|l|l|l|}
			\hline
			Data & Numer wersji & Opis & Autor \\
			\hline \hline
			2018-11-26 & 0.1 & Utworzenie dokumentu & Piotr Kowalczyk \\
			\hline 
			2018-11-30 & 0.2 & Korekta dokumentu & Agnieszka Dudek \\
			\hline
    \end{tabular}
	\end{center}
\end{table}	

\tableofcontents

\newpage


\section{Wprowadzenie}

\subsection{Cel dokumentu}
Niniejszy dokument ma na celu precyzyjne określenie wymagań projektu \textit{Tablica efektów kształcenia}. Stworzony jest na potrzeby przedmiotu inżynieria oprogramowania.


\section{Historyjki użytkownika}
\subsection{Pierwsza historyjka}
Jako student informatyki, który właśnie układa swój plan zajęć, chcę sprawdzić, ile punktów ECTS z danych grup przedmiotów już zdobyłem. Dzięki temu w~kolejnym semestrze 
będę mógł wybrać takie przedmioty, które uzupełnią punkty brakujące do wymaganego limitu.
\subsection{Druga historyjka}
Jako studentka informatyki, która zaczyna ostatni semestr studiów, chcę zobaczyć, jakich efektów kształcenia jeszcze nie osiągnęłam i
upewnić się, że wszystkie wymagania spełnię w ostatnim półroczu i na czas będę mogła skończyć studia.
\subsection{Trzecia historyjka}
Jako osoba zatrudniona w dziekanacie dowiaduję się, że student wypracował zaliczenie z przedmiotu algorytmy 
i struktury danych w lutym, prosi zatem o~przedłużenie sesji oraz korektę danych. 
Chcę, aby po zmianie oceny oraz efektów kształcenia w~USOSie zauktualizowane zostały dane nt. efektów kształcenia w~Systemie Zapisów, aby nie było konieczności ręcznego poprawiania danych.


\section{Wymagania funkcjonalne}

\subsection{Pokazanie \textit{Tablicy efektów} użytkownikowy Systemu Zapisów}
Jest to kluczowa funkcja projektowanego systemu. Po zleceniu systemowi wyświetlenia \textit{Tablicy efektów} użytkownik powienien niezwłocznie zobaczyć:
\begin{itemize}
	\item listę osiągniętych efektów kształcenia,
	\item listę nieosiągniętych efektów koniecznych do ukończenia aktualnego etapu studiów,
	\item punkty ECTS zdobyte z: przedmiotów humanistycznych oraz informatycznych, projektów i kursów.
\end{itemize}

\subsection{Generowanie danych do pliku}
Drobną, lecz często bardzo przydatną funkcjonalnością będzie możliwość wygenerowania \textit{Tablicy efektów} do pliku w formacie pdf oraz wydrukowania jej.
Będzie to użyteczne zarówno dla studentów, którzy wolą analizować dane na papierze niż na komputerze, jak i dla pań z dziekanatu, do których jeszcze przez jakiś czas będą przychodzić studenci z prośbą o dostarczenie
wiarygodnych i kompletnych informacji o osiągniętych efektach.

\subsection{Importowanie danych z USOSa}
Ręczne wpisywanie danych wszystkich studentów korzystających z Systemu Zapisów byłoby zaprzeczeniem wszystkiego, w co wierzymy. Projektowany system musi automatycznie synchronizować dane dotyczące efektów kształcenia studentów z USOSem.


\subsection{Ręczna modyfikacja danych}
Zdarza się, że student osiąga któryś z efektów kształcenia zaliczając przedmiot po terminie bądź bez uczęszczania na jakikolwiek przedmiot (np. możliwe jest zwolnienie z projektu zespołowego na podstawie pracy zawodowej). Zdarzają się także błędy nawet w najdoskonalszych systemach informatycznych, a USOSa trudno byłoby zaliczyć do tej grupy. Konieczna jest zatem możliwość ręcznej modyfikacji danych, które znajdują się w Systemie Zapisów i dotyczą efektów kształcenia.



\section{Wymagania niefunkcjonalne}

\subsection{Płynność działania}
Przez większość czasu z Systemu Zapisów korzysta niewielka liczba studentów, jednak zmienia się to po otwarciu zapisów na dany semestr oraz w trakcie głosowania. W tych okresach Tablica efektów kształcenia będzie szczególnie często użytkowana.
\\ Aby sprawdzić, czy na pewno program działa płynnie, należy przetestować system wysyłając do niego 600 zapytań w krótkim czasie. System powinien działać w każdym przypadku, nawet gdyby wszyscy studenci chcieli z niego korzystać w tym samym momencie, a teraz 
na naszej uczelni na kierunku informatyka studiuje 525 osób.

\subsection{Przejrzystość}
Wyświetlane przez nas dane mają studentom przede wszystkim pomóc. Zależy nam na tym, aby były przedstawione w sposób przejrzysty i zrozumiały.
\\Aby upewnić się, że rzeczywiście tak jest, należy skonsultować ze studentami i~pracownikami dziekanatu proponowany format przedstawianych informacji.

\end{document}
