\documentclass{article}
\usepackage{multimedia}
\usepackage{polski}
\usepackage[utf8]{inputenc}
\usepackage[T1]{fontenc}
\usepackage{color}
\usepackage{amsthm}
\usepackage[labelsep=period]{caption}

\renewcommand*{\tablename}{Tabela}
\renewcommand{\thetable}{\arabic{figure}}
\renewcommand*{\thesection}{\arabic{section}.}
\renewcommand*{\thesubsection}{\arabic{section}.\arabic{subsection}.}


\begin{document}

\begin{titlepage}
	\vspace{2em}{\centering\large{Studencka Pracownia Inżynierii Oprogramowania}\par}

	\vspace{2em}{\centering\large{Instytut Informatyki Uniwersytetu Wrocławskiego}\par}

  \vspace{16em}{\centering\large{Agnieszka Dudek, Piotr Kowalczyk}\par}

	\vspace{2em}{\centering\huge{Dokumentacja projektu \textit{Tablica efektów kształcenia}}\par}

	\vspace{4em}{\centering\Large{Wprowadzenie na rynek}\par}

	\vspace{16em}{\centering\large{Wrocław, \today}\par}

	\vspace{1em}{\centering\large{Wersja 0.1}\par}

\end{titlepage}

\addtocounter{page}{1}
\newpage

\begin{table}[h!]
 \begin{center}
	\caption{Historia zmian dokonywanych w dokumencie}
   \begin{tabular}{|l|l|l|l|}
		\hline
		Data & Numer wersji & Opis & Autor \\
		\hline \hline
		2018-12-19 & 0.1 & Utworzenie dokumentu & Piotr Kowalczyk \\
		\hline
  \end{tabular}
 \end{center}
\end{table}	


\tableofcontents

\newpage


\section{Wprowadzenie}
\subsection{Cel dokumentu}
Niniejszy dokument został stworzony w celu zaplanowania wprowadzenia wykonanego projektu na rynek.
Dokument stanowi styczniowe zadanie realizowane w ramach pracowni z inżynierii oprogramowania.


\section{Plan wdrożenia}
\textit{Tablica efektów} stanie się częścią Systemu Zapisów.
Ten fakt niesamowicie upraszcza wdrożenie, które, ponieważ \textit{Tablica efektów} nie wpływa na dotychczasowe działanie Systemu Zapisów, ograniczy się do uruchomienia się nowszej wersji Systemu Zapisów.

TODO: kalendarz z pierwszą migracją oraz szkoleniami.

\section{Organizacja szkoleń użytkowników}

\subsection{Szkolenie pracowników dziekanatu}
TODO

\subsection{Szkolenie studentów}
Szkolenie studentów nie będzie wymagane, ponieważ produkt stanie się częścią Systemu Zapisów, który studenci potrafią obsługiwać, oraz dlatego, że obsługa ze strony studenta sprowadza się do wyświetlenia tablicy oraz wyeksportowania jej do pliku pdf.

\section{Wsparcie techniczne}
Podczas wdrożenia oraz testów beta nasi programiści będą się komunikować z testerami oraz osobami szkolonymi w celu identyfikacji błędów oraz kolekcjonowania skarg i zażaleń.

TODO: napisać to na serio

\section{Sformułowanie głównych punktów umów}

\section{Zaproponowanie sposobu pomiaru satysfakcji klienta}
Dołożymy do oceny zajęć na następny semestr punkt "Dlaczego podoba Ci się Twoja nowa tablica efektów? Napisz rozprawkę na zadany temat. Określ swoje stanowisko i poprzyj je argumentacją. Odnieś się do co najmniej trzech tekstów kultury."

I jakaś ankietka dla pracowników.

TODO: to też napisać na serio


\end{document}
