\documentclass{article}
\usepackage{multimedia}
\usepackage{polski}
\usepackage[utf8]{inputenc}
\usepackage[T1]{fontenc}
\usepackage{color}
\usepackage{amsthm}
\usepackage[labelsep=period]{caption}

\renewcommand*{\tablename}{Tabela}
\renewcommand{\thetable}{\arabic{figure}}
\renewcommand*{\thesection}{\arabic{section}.}
\renewcommand*{\thesubsection}{\arabic{section}.\arabic{subsection}.}

\newcommand{\TaE}{\textit{Tablica Efektów}}
\newcommand{\TyE}{\textit{Tablicy Efektów}}
\newcommand{\TeE}{\textit{Tablicę Efektów}}

\begin{document}

\begin{titlepage}
	\vspace{2em}{\centering\large{Studencka Pracownia Inżynierii Oprogramowania}\par}

	\vspace{2em}{\centering\large{Instytut Informatyki Uniwersytetu Wrocławskiego}\par}

  \vspace{16em}{\centering\large{Agnieszka Dudek, Piotr Kowalczyk}\par}

	\vspace{2em}{\centering\huge{Dokumentacja projektu \textit{Tablica efektów kształcenia}}\par}

	\vspace{4em}{\centering\Large{Wprowadzenie na rynek}\par}

	\vspace{16em}{\centering\large{Wrocław, \today}\par}

	\vspace{1em}{\centering\large{Wersja 0.2}\par}

\end{titlepage}

\addtocounter{page}{1}
\newpage

\begin{table}[h!]
 \begin{center}
	\caption{Historia zmian dokonywanych w dokumencie}
   \begin{tabular}{|l|l|l|l|}
		\hline
		Data & Numer wersji & Opis & Autor \\
		\hline \hline
		2018-12-19 & 0.1 & Utworzenie dokumentu & Piotr Kowalczyk \\
		\hline
		2019-01-27 & 0.2 & Korekta dokumentu & Agnieszka Dudek \\
		\hline
		
  \end{tabular}
 \end{center}
\end{table}	


\tableofcontents

\newpage


\section{Wprowadzenie}
\subsection{Cel dokumentu}
Niniejszy dokument został stworzony w celu zaplanowania wprowadzenia wykonanego projektu na rynek.
Dokument stanowi styczniowe zadanie realizowane w ramach pracowni z inżynierii oprogramowania.


\section{Plan wdrożenia}
\textit{Tablica efektów} stanie się częścią Systemu Zapisów.
Ten fakt niesamowicie upraszcza wdrożenie, które, ponieważ \textit{Tablica efektów} nie wpływa na dotychczasowe działanie Systemu Zapisów, ograniczy się do uruchomienia się nowszej wersji Systemu Zapisów.

Pierwsza migracja danych zaplanowana jest na 1 sierpnia - wtedy też zaczną się testy beta.
Równocześnie będą się odbywały szkolenia pracowników, a wprowadzone zmiany zostaną przedstawione studentom.
Założymy też post na Forum II UWr, pod którym studenci będą mogli wpisywać swoje uwagi do \textit{Tablicy efektów}.

\section{Organizacja szkoleń użytkowników}

\subsection{Szkolenie pracowników dziekanatu}
Dzięki \TyE\ dane studentów z systemu USOS będą importowane automatycznie do Systemu Zapisów. Ważne jest przy tym, żeby pracownicy dziekanatu poznali wszystkie 
wyświetlane przez \TeE\ dane i w razie wątpliwości studenta mogli je szczegółowo wyjaśnić. Na przykładach historyjek użytkowania przeanalizujemy z pracownikami wszystkie
elementy, a następnie zostawimy broszury przypominające z objaśnionymi polami \TyE.

\subsection{Szkolenie studentów}
Szkolenie studentów nie będzie wymagane, ponieważ produkt stanie się częścią Systemu Zapisów, który studenci potrafią obsługiwać, oraz dlatego, że obsługa ze strony studenta sprowadza się do wyświetlenia tablicy oraz wyeksportowania jej do pliku PDF.

\section{Wsparcie techniczne}
Wdrożenie nowej wersji Systemu Zapisów będzie miało miejsce w Instytucie Informatyki we współpracy z osobami odpowiedzialnymi za System Zapisów.
Podczas wdrożenia oraz testów beta nasi programiści będą na bieżąco komunikowali się z testerami i starali jak najszybciej naprawiać znalezione błędy.
Dodatkowo osoba odpowiedzialna za szkolenia będzie w kontakcie z pracownikami dziekanatu w razie potrzebnych konsultacji.

\section{Sformułowanie głównych punktów umów}
\textit{Tablica efektów} będzie powstawała w ramach przedmiotu \textit{Rozwój Systemu Zapisów}, dlatego nie ma potrzeby ani możliwości podpisywania umów ze studentami.

\section{Zaproponowanie sposobu pomiaru satysfakcji klienta}
Do cosemestralnej oceny zajęć dodamy punkt "Czy podoba Ci się Twoja nowa \TaE? Określ swoje stanowisko i poprzyj je argumentacją." Spotkamy się również z pracownikami dziekanatu, 
aby mogli podzielić się z nami swoimi uwagami do \TyE. 

Będziemy też śledzić i analizować liczbę wejść w zakładkę \TaE\ w~Systemie Zapisów przez studentów w ciągu roku akademickiego, w czasie sesji i wakacji.

\end{document}
