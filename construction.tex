\documentclass{article}
\usepackage{multimedia}
\usepackage{polski}
\usepackage[utf8]{inputenc}
\usepackage[T1]{fontenc}
\usepackage{color}
\usepackage{amsthm}
\usepackage[labelsep=period]{caption}

\renewcommand*{\tablename}{Tabela}
\renewcommand{\thetable}{\arabic{figure}}
\renewcommand*{\thesection}{\arabic{section}.}
\renewcommand*{\thesubsection}{\arabic{section}.\arabic{subsection}.}


\begin{document}

\begin{titlepage}
	\vspace{2em}{\centering\large{Instytut Informatyki Uniwersytetu Wrocławskiego}\par}

  \vspace{17em}{\centering\large{Agnieszka Dudek, Piotr Kowalczyk}\par}

	\vspace{2em}{\centering\huge{Dokumentacja projektu \textit{Tablica efektów kształcenia}}\par}

	\vspace{4em}{\centering\Large{Konstrukcja projektu}\par}

	\vspace{17em}{\centering\large{Wrocław, \today}\par}

	\vspace{1em}{\centering\large{Wersja 0.1}\par}

\end{titlepage}

\addtocounter{page}{1}
\newpage

\begin{table}[h!]
	\begin{center}
		\caption{Historia zmian dokonywanych w dokumencie}
		\begin{tabular}{|l|l|l|l|}
			\hline
			Data & Numer wersji & Opis & Autor \\
			\hline \hline
			2018-12-19 & 0.1 & Utworzenie dokumentu & Agnieszka Dudek \\
			\hline 
    \end{tabular}
	\end{center}
\end{table}	

\tableofcontents

\newpage


\section{Wprowadzenie}

\subsection{Cel dokumentu}
Niniejszy dokument ma na celu TODO
Dokument stanowi grudniowe zadanie realizowane w ramach pracowni z inżynierii oprogramowania.


\section{Testy funkcjonalne dla historyjek użytkowania}
\subsection{Pierwsza historyjka}
Jako student informatyki, który właśnie układa swój plan zajęć, chcę sprawdzić, ile punktów ECTS z danych grup przedmiotów już zdobyłem.
Dzięki temu w kolejnym semestrze będę mógł wybrać takie przedmioty, które uzupełnią punkty brakujące do wymaganego limitu.

TODO
\subsection{Druga historyjka}
Jako studentka informatyki, która zaczyna ostatni semestr studiów, chcę zobaczyć, jakich efektów kształcenia jeszcze nie osiągnęłam i upewnić się, że wszystkie wymagania spełnię w ostatnim półroczu i na czas będę mogła skończyć studia.

TODO
\subsection{Trzecia historyjka}
Jako osoba zatrudniona w dziekanacie dowiaduję się, że student wypracował zaliczenie z przedmiotu \textit{Algorytmy i struktury danych} w lutym, prosi zatem o przedłużenie sesji oraz korektę danych. 
Chcę, aby po zmianie oceny oraz efektów kształcenia w USOS zauktualizowane zostały dane dotyczące efektów kształcenia w Systemie Zapisów, aby nie było konieczności ręcznego poprawiania danych.

TODO

\section{Zgodność względem norm ISO/IEC 9126 i 25000}

TODO
\subsection{}

\section{Plan beta testowania}
TODO

\section{Plan zarządzania ryzykiem}
TODO

\section{Szczegółowe plany}
\subsection{Plan wykonania produktu}
TODO
\subsection{Ocena pracochłonności}
TODO
\subsection{Harmonogram}
TODO

\section{Ocena zgodności wykonanych prac z wizją systemu i specyfikacją wymagań}
TODO

\end{document}
