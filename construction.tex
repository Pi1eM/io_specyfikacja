\documentclass{article}
\usepackage{multimedia}
\usepackage{polski}
\usepackage[utf8]{inputenc}
\usepackage[T1]{fontenc}
\usepackage{color}
\usepackage{amsthm}
\usepackage[labelsep=period]{caption}

\renewcommand*{\tablename}{Tabela}
\renewcommand{\thetable}{\arabic{figure}}
\renewcommand*{\thesection}{\arabic{section}.}
\renewcommand*{\thesubsection}{\arabic{section}.\arabic{subsection}.}


\begin{document}

\begin{titlepage}
	\vspace{2em}{\centering\large{Studencka Pracownia Inżynierii Oprogramowania}\par}

	\vspace{2em}{\centering\large{Instytut Informatyki Uniwersytetu Wrocławskiego}\par}

  \vspace{17em}{\centering\large{Agnieszka Dudek, Piotr Kowalczyk}\par}

	\vspace{2em}{\centering\huge{Dokumentacja projektu \textit{Tablica efektów kształcenia}}\par}

	\vspace{4em}{\centering\Large{Konstrukcja projektu}\par}

	\vspace{15.5em}{\centering\large{Wrocław, \today}\par}

	\vspace{1em}{\centering\large{Wersja 0.7}\par}

\end{titlepage}

\addtocounter{page}{1}
\newpage

\begin{table}[h!]
	\begin{center}
		\caption{Historia zmian dokonywanych w dokumencie}
		\begin{tabular}{|l|l|l|l|}
			\hline
			Data & Numer wersji & Opis & Autor \\
			\hline \hline
			2018-12-19 & 0.1 & Utworzenie dokumentu & Agnieszka Dudek \\
			\hline
			2019-01-04 & 0.2 & Korekta dokumentu & Agnieszka Dudek \\
			\hline
			2019-01-08 & 0.3 & Korekta dokumentu & Piotr Kowalczyk \\
			\hline 
			2019-01-09 & 0.4 & Korekta dokumentu & Agnieszka Dudek \\
			\hline 
			2019-01-15 & 0.5 & Korekta dokumentu & Agnieszka Dudek \\
			\hline 
			2019-01-16 & 0.6 & Korekta dokumentu & Piotr Kowalczyk \\
			\hline 
			2019-01-21 & 0.7 & Korekta dokumentu & Agnieszka Dudek \\
			\hline 


    \end{tabular}
	\end{center}
\end{table}	

\tableofcontents

\newpage


\section{Wprowadzenie}

\subsection{Cel dokumentu}
Niniejszy dokument ma na celu dokładne przygotowanie testów oraz zaplanowanie zarządzania jakością i ryzykiem programu.
Dokument stanowi grudniowe zadanie realizowane w ramach pracowni z inżynierii oprogramowania.


\section{Testy funkcjonalne dla historyjek użytkowania}
\subsection{Pierwsza historyjka}
\textit{Jako student informatyki, który właśnie układa swój plan zajęć, chcę sprawdzić, ile punktów ECTS z danych grup przedmiotów już zdobyłem.
Dzięki temu w kolejnym semestrze będę mógł wybrać takie przedmioty, które uzupełnią punkty brakujące do wymaganego limitu.}

\medskip
\noindent Kroki potrzebne do wykonania testu:
\begin{itemize}
 \item tworzymy profil studenta testowego,
 \item w zewnętrznym systemie zapisujemy studenta na trzy kursy: \textit{Analiza numeryczna(L)}, \textit{Programowanie obiektowe}, \textit{Przedmiot humanistyczny}, a następnie zaznaczamy, że student zaliczył te przedmioty,
 \item w zakładce \textit{Tablica efektów} sprawdzamy, czy wszystkie trzy przedmioty są wpisane w odpowiednich kolumnach oraz czy zgadza się liczba punktów ECTS.
\end{itemize}


\subsection{Druga historyjka}
\textit{Jako studentka informatyki, która zaczyna ostatni semestr studiów, chcę zobaczyć, jakich efektów kształcenia jeszcze nie osiągnęłam i upewnić się, że wszystkie wymagania spełnię w ostatnim półroczu i na czas będę mogła skończyć studia.}

\medskip
\noindent Kroki potrzebne do wykonania testu:
\begin{itemize}
 \item tworzymy profil studenta testowego,
 \item w zewnętrznym systemie zapisujemy studenta na trzy kursy: \textit{Analiza numeryczna(L)}, \textit{Programowanie obiektowe}, \textit{Przedmiot humanistyczny}, a następnie zaznaczamy, że student zaliczył te przedmioty,
 \item w zakładce \textit{Tablica efektów} sprawdzamy, czy w przedmiotach niezrealizowanych znajdują się wszystkie wymagane w programie studiów pozostałe przedmioty oraz
 liczba brakujących punktów ECTS się zgadza.
\end{itemize}

\subsection{Trzecia historyjka}
\textit{Jako osoba zatrudniona w dziekanacie dowiaduję się, że student wypracował zaliczenie z przedmiotem \textnormal{Algorytmy i struktury danych} w lutym, prosi zatem o przedłużenie sesji oraz korektę danych. 
Chcę, aby po zmianie oceny oraz efektów kształcenia w USOS zaktualizowane zostały dane dotyczące efektów kształcenia w Systemie Zapisów, aby nie było konieczności ręcznego poprawiania danych.}

\medskip
\noindent Kroki potrzebne do wykonania testu:
\begin{itemize}
 \item tworzymy profil studenta testowego,
 \item w zewnętrznym systemie zapisujemy studenta na kurs \textit{Algorytmy i struktury danych} i zaznaczamy jako przedmiot niezdany,
 \item w zakładce \textit{Tablica efektów} sprawdzamy, że przedmiot \textit{Algorytmy i struktury danych} jest w kolumnie przedmiotów niezrealizowanych,
 \item w zewnętrznym systemie zaznaczamy teraz, że kurs \textit{Algorytmy i struktury danych} został zaliczony,
 \item w zakładce \textit{Tablica efektów} sprawdzamy, że przedmiot \textit{Algorytmy i struktury danych} już jest w kolumnie przedmiotów zrealizowanych.
\end{itemize}

\section{Opracowanie pomiarów sprawdzających wymagania niefunkcjonalne zgodnych z normami \\ISO/IEC 9126 i 5000}
Model jakościowy dostarczony przez standard ISO/IEC 9126-1 definiuje jakość \textit{Tablicy efektów} jako zbiór wielu cech, spośród których największy nacisk kładziemy na płynność działania (wydajność), przejrzystość oraz niezawodność.

W celu przetestowania płynności działania \textit{Tablicy efektów} wyślemy do niej 600 zapytań w ciągu minuty.
Odpowiedź powinna trwać nie dłużej niż czterokrotność średniego czasu odpowiedzi \textit{Tablicy efektów} (czas ten zależy także od szybkości i obciążenia łącza, zatem musi być mierzony bezpośrednio przed testem).

Przejrzystość interfejsu będzie oceniana przez testerów beta.

W celu przetestowania niezawodności będziemy zbierać informacje dotyczące poszczególnych zapytań.
\textit{Tablica efektów} będzie działała poprawnie, jeśli zwróci nie więcej niż jedną niepoprawną odpowiedź na 100 poprawnych zapytań oraz jeśli nie nastąpi więcej niż jeden krytyczny (wymagający ingerencji administratora w celu dalszego działania) błąd na 10000 zapytań (niekoniecznie poprawnych).

\newpage

\section{Plan testowania beta}
Przewidujemy dostarczyć projekt na początek semestru zimowego 2019/2020.
Aby dobrze przetestować \textit{Tablicę efektów}, udostępnimy program studentom drugiego i trzeciego roku oraz pracownikom dziekanatu na okres wakacyjny.
Dzięki temu w razie usterek będziemy mieli czas na naprawienie błędów.
Wtedy też studenci nie będą podejmowali żadnych ostatecznych decyzji (związanych z głosowaniem, wyborem przedmiotów), zatem będzie to dobry czas na testy.

\section{Plan uwzględniania ryzyka}
Jak wspomniano w dokumencie \textit{Koncepcja wykonania systemu}, największe ryzyko dotyczące \textit{Tablicy efektów} to ryzyko opóźnień w harmonogramie.
Aby temu przeciwdziałać, postaramy się rozłożyć zadania programistyczne i testowe z naciskiem na pierwszą połowę czasu ustalonego w harmonogramie (p. 6.2).
Ponieważ ostateczny termin wdrożenia \textit{Tablicy efektów} to połowa września (wtedy zdecydowana większość studentów zacznie interesować się zapisami na kolejny semestr), mamy dodatkowy miesiąc zapasu na ewentualne opóźnienia.

\section{Szczegółowe plany}
\subsection{Plan wykonania produktu}
Na początku nawiązany zostanie kontakt z osobami odpowiedzialnymi za rozwój Systemu Zapisów.
Po ich akceptacji, omówieniu ich roli w trakcie wykonywania projektu oraz wyznaczeniu programistów do realizacji, rozpocznie się proces wytwarzania oprogramowania.
Będzie on zgodny ze standardem używanym w Systemie Zapisów (zawiera m.in. przegląd kodu), uzupełnionym przez dodatkowego testera.

Po wytworzeniu \textit{Tablicy efektów} nastąpią dwie fazy testów.
Podczas pierwszej fazy będzie sprawdzona zgodność \textit{Tablicy efektów} z Systemem Zapisów, jego współdziałanie z USOS, a także wydajność i inne wymagania niefunkcjonalne.
Po potwierdzeniu poprawności działania \textit{Tablicy efektów} przez ww. testy, rozpocznie się faza testowania beta (p. 4).
W trakcie testowania błędy będą poprawiane zaraz po wykryciu.

Po zakończeniu testów osoby pracujące w dziekanacie zostaną przeszkolone do używania \textit{Tablicy efektów}.
W tym samym czasie \textit{Tablicy efektów} będzie wdrażana.

\subsection{Harmonogram i ocena pracochłonności}
Harmonogram jest następujący (wszystkie daty dotyczą roku 2019):
\begin{samepage}
\begin{itemize}
 \item 1 lutego -- 20 lutego --- omówienie projektu z osobami odpowiedzialnymi za System Zapisów,
 \item 21 lutego -- 31 maja --- wytwarzanie oprogramowania,
 \item 1 czerwca -- 30 czerwca --- testy, faza pierwsza,
 \item 1 lipca -- 31 lipca --- faza testowania beta,
 \item 1 sierpnia -- 14 sierpnia --- szkolenia użytkowników i wdrożenie \textit{Tablicy efektów}.
\end{itemize}
\end{samepage}
Niezależnie od powyższego harmonogramu, \textit{Tablica efektów} zacznie być używana przez klientów w połowie września (wtedy studenci zaczynają się interesować ofertą zajęć na kolejny semestr).
Zatem czas od 15 sierpnia do 14 września jest dodatkowym buforem, który zostanie wykorzystany w razie opóźnień.

\section{Ocena zgodności wykonanych prac z wizją systemu i specyfikacją wymagań}
Zgodnie z głównym założeniem dotyczącym \textit{Tablicy efektów} dane wyświetlane studentom muszą być kompletne i przydatne.
Będzie to oceniane w trakcie testów beta, kiedy to studenci oraz pracownicy dziekanatu będą zachęcani do wyrażenia swojej opinii.

Wizja wyglądu \textit{Tablicy efektów} nie jest formalnie ustalona.
Dopuszczamy wszelkie zmiany, pod warunkiem że będą się one komponować wizualnie z istniejącymi już interfejsami Systemu Zapisów.


\end{document}
