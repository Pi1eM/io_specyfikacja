\documentclass{article}
\usepackage{multimedia}
\usepackage{polski}
\usepackage[utf8]{inputenc}
\usepackage[T1]{fontenc}
\usepackage{color}
\usepackage{amsthm}
\usepackage[labelsep=period]{caption}

\renewcommand*{\tablename}{Tabela}
\renewcommand{\thetable}{\arabic{figure}}
\renewcommand*{\thesection}{\arabic{section}.}
\renewcommand*{\thesubsection}{\arabic{section}.\arabic{subsection}.}


\begin{document}

\begin{titlepage}
	\vspace{2em}{\centering\large{Instytut Informatyki Uniwersytetu Wrocławskiego}\par}

  \vspace{17em}{\centering\large{Agnieszka Dudek, Piotr Kowalczyk}\par}

	\vspace{2em}{\centering\huge{Dokumentacja projektu \textit{Tablica efektów kształcenia}}\par}

	\vspace{4em}{\centering\Large{Konstrukcja projektu}\par}

	\vspace{17em}{\centering\large{Wrocław, \today}\par}

	\vspace{1em}{\centering\large{Wersja 0.2}\par}

\end{titlepage}

\addtocounter{page}{1}
\newpage

\begin{table}[h!]
	\begin{center}
		\caption{Historia zmian dokonywanych w dokumencie}
		\begin{tabular}{|l|l|l|l|}
			\hline
			Data & Numer wersji & Opis & Autor \\
			\hline \hline
			2018-12-19 & 0.1 & Utworzenie dokumentu & Agnieszka Dudek \\
			\hline \hline
			2019-01-4 & 0.2 & Korekta dokumentu & Agnieszka Dudek \\
			\hline 
    \end{tabular}
	\end{center}
\end{table}	

\tableofcontents

\newpage


\section{Wprowadzenie}

\subsection{Cel dokumentu}
Niniejszy dokument ma na celu dokładne przygotowanie testów oraz zaplanowanie zarządzania jakością i ryzkiem programu.
Dokument stanowi grudniowe zadanie realizowane w ramach pracowni z inżynierii oprogramowania.


\section{Testy funkcjonalne dla historyjek użytkowania}
\subsection{Pierwsza historyjka}
\textit{Jako student informatyki, który właśnie układa swój plan zajęć, chcę sprawdzić, ile punktów ECTS z danych grup przedmiotów już zdobyłem.
Dzięki temu w kolejnym semestrze będę mógł wybrać takie przedmioty, które uzupełnią punkty brakujące do wymaganego limitu.}

\medskip
\noindent Kroki potrzebne do przetestowania funkcjonalności:
\begin{itemize}
 \item tworzymy profil studenta testowego,
 \item w zewnętrznym systemie zapisujemy studenta na 3 kursy: \textit{Analiza numeryczna(L)}, \textit{Programowanie Obiektowe}, \textit{Przedmiot humanistyczny}, a następnie zaznaczamy, że student zaliczył te przedmioty,
 \item w zakładce \textit{Tablica efektów} sprawdzamy, czy wszystkie 3 przedmioty są wpisane w odpowiednich kolumnach oraz czy zgadza się liczba punktów ECTS.
\end{itemize}


\subsection{Druga historyjka}
\textit{Jako studentka informatyki, która zaczyna ostatni semestr studiów, chcę zobaczyć, jakich efektów kształcenia jeszcze nie osiągnęłam i upewnić się, że wszystkie wymagania spełnię w ostatnim półroczu i na czas będę mogła skończyć studia.}

\medskip
\noindent Kroki potrzebne do przetestowania funkcjonalności:
\begin{itemize}
 \item tworzymy profil studenta testowego,
 \item w zewnętrznym systemie zapisujemy studenta na 3 kursy: \textit{Analiza numeryczna(L)}, \textit{Programowanie Obiektowe}, \textit{Przedmiot humanistyczny}, a następnie zaznaczamy, że student zaliczył te przedmioty,
 \item w zakładce \textit{Tablica efektów} sprawdzamy, czy w przedmiotach niezrealizowanych znajdują się wszystkie wymagane w programie studiów pozostałe przedmioty oraz
 liczba brakujących punktów ECTS się zgadza.
\end{itemize}

\subsection{Trzecia historyjka}
\textit{Jako osoba zatrudniona w dziekanacie dowiaduję się, że student wypracował zaliczenie z przedmiotu \textit{Algorytmy i struktury danych} w lutym, prosi zatem o przedłużenie sesji oraz korektę danych. 
Chcę, aby po zmianie oceny oraz efektów kształcenia w USOS zauktualizowane zostały dane dotyczące efektów kształcenia w Systemie Zapisów, aby nie było konieczności ręcznego poprawiania danych.}

\medskip
\noindent Kroki potrzebne do przetestowania funkcjonalności:
\begin{itemize}
 \item tworzymy profil studenta testowego,
 \item w zewnętrznym systemie zapisujemy studenta na kurs \textit{Algorytmy i struktury danych} i zaznaczamy jako przedmiot niezdany,
 \item w zakładce \textit{Tablica efektów} sprawdzamy, że przedmiot \textit{Algorytmy i struktury danych} jest w kolumnie przedmiotów niezrealizowanych,
 \item w zewnętrznym systemie zaznaczamy teraz, że kurs \textit{Algorytmy i struktury danych} został zaliczony,
 \item w zakładce \textit{Tablica efektów} sprawdzamy, że przedmiot \textit{Algorytmy i struktury danych} już jest w kolumnie przedmiotów zrealizowanych.
\end{itemize}

\section{Zgodność względem norm ISO/IEC 9126 i 25000}
TODO

\section{Plan beta testowania}
Przewidujemy dostarczyć projekt na początek semestru zimowego 2019/2020. Aby dobrze przetestować \textit{Tablicę efektów}, udostępnimy program studentom drugiego
i trzeciego roku oraz pracownikom dziekanatu na okres wakacyjny. Dzięki temu w razie usterek będziemy mieć czas na naprawienie błędów, w tym czasie też studenci nie podejmują żadnych ostatecznych decyzji (np. związanych z głosowaniem, wyborem przedmiotów).

\section{Plan zarządzania ryzykiem}
TODO

\section{Szczegółowe plany}
\subsection{Plan wykonania produktu}
TODO
\subsection{Ocena pracochłonności}
TODO
\subsection{Harmonogram}
TODO

\section{Ocena zgodności wykonanych prac z wizją systemu i specyfikacją wymagań}
TODO

\end{document}
