\documentclass{article}
\usepackage{multimedia}
\usepackage{polski}
\usepackage[utf8]{inputenc}
\usepackage[T1]{fontenc}
\usepackage{color}
\usepackage{amsthm}
\usepackage[labelsep=period]{caption}

\renewcommand*{\tablename}{Tabela}
\renewcommand{\thetable}{\arabic{figure}}
\renewcommand*{\thesection}{\arabic{section}.}
\renewcommand*{\thesubsection}{\arabic{section}.\arabic{subsection}.}


\begin{document}

\begin{titlepage}
	\vspace{2em}{\centering\large{Instytut Informatyki Uniwersytetu Wrocławskiego}\par}

  \vspace{17em}{\centering\large{Agnieszka Dudek, Piotr Kowalczyk}\par}

	\vspace{2em}{\centering\huge{Dokumentacja projektu \textit{Tablica efektów kształcenia}}\par}

	\vspace{4em}{\centering\Large{Konstrukcja projektu}\par}

	\vspace{17em}{\centering\large{Wrocław, \today}\par}

	\vspace{1em}{\centering\large{Wersja 0.3}\par}

\end{titlepage}

\addtocounter{page}{1}
\newpage

\begin{table}[h!]
	\begin{center}
		\caption{Historia zmian dokonywanych w dokumencie}
		\begin{tabular}{|l|l|l|l|}
			\hline
			Data & Numer wersji & Opis & Autor \\
			\hline \hline
			2018-12-19 & 0.1 & Utworzenie dokumentu & Agnieszka Dudek \\
			\hline
			2019-01-04 & 0.2 & Korekta dokumentu & Agnieszka Dudek \\
			\hline
			2019-01-08 & 0.3 & Korekta dokumentu & Piotr Kowalczyk \\
			\hline 
    \end{tabular}
	\end{center}
\end{table}	

\tableofcontents

\newpage


\section{Wprowadzenie}

\subsection{Cel dokumentu}
Niniejszy dokument ma na celu dokładne przygotowanie testów oraz zaplanowanie zarządzania jakością i ryzkiem programu.
Dokument stanowi grudniowe zadanie realizowane w ramach pracowni z inżynierii oprogramowania.


\section{Testy funkcjonalne dla historyjek użytkowania}
\subsection{Pierwsza historyjka}
\textit{Jako student informatyki, który właśnie układa swój plan zajęć, chcę sprawdzić, ile punktów ECTS z danych grup przedmiotów już zdobyłem.
Dzięki temu w kolejnym semestrze będę mógł wybrać takie przedmioty, które uzupełnią punkty brakujące do wymaganego limitu.}

\medskip
\noindent Kroki potrzebne do przetestowania funkcjonalności:
\begin{itemize}
 \item tworzymy profil studenta testowego,
 \item w zewnętrznym systemie zapisujemy studenta na 3 kursy: \textit{Analiza numeryczna(L)}, \textit{Programowanie Obiektowe}, \textit{Przedmiot humanistyczny}, a następnie zaznaczamy, że student zaliczył te przedmioty,
 \item w zakładce \textit{Tablica efektów} sprawdzamy, czy wszystkie 3 przedmioty są wpisane w odpowiednich kolumnach oraz czy zgadza się liczba punktów ECTS.
\end{itemize}


\subsection{Druga historyjka}
\textit{Jako studentka informatyki, która zaczyna ostatni semestr studiów, chcę zobaczyć, jakich efektów kształcenia jeszcze nie osiągnęłam i upewnić się, że wszystkie wymagania spełnię w ostatnim półroczu i na czas będę mogła skończyć studia.}

\medskip
\noindent Kroki potrzebne do przetestowania funkcjonalności:
\begin{itemize}
 \item tworzymy profil studenta testowego,
 \item w zewnętrznym systemie zapisujemy studenta na 3 kursy: \textit{Analiza numeryczna(L)}, \textit{Programowanie Obiektowe}, \textit{Przedmiot humanistyczny}, a następnie zaznaczamy, że student zaliczył te przedmioty,
 \item w zakładce \textit{Tablica efektów} sprawdzamy, czy w przedmiotach niezrealizowanych znajdują się wszystkie wymagane w programie studiów pozostałe przedmioty oraz
 liczba brakujących punktów ECTS się zgadza.
\end{itemize}

\subsection{Trzecia historyjka}
\textit{Jako osoba zatrudniona w dziekanacie dowiaduję się, że student wypracował zaliczenie z przedmiotu \textit{Algorytmy i struktury danych} w lutym, prosi zatem o przedłużenie sesji oraz korektę danych. 
Chcę, aby po zmianie oceny oraz efektów kształcenia w USOS zauktualizowane zostały dane dotyczące efektów kształcenia w Systemie Zapisów, aby nie było konieczności ręcznego poprawiania danych.}

\medskip
\noindent Kroki potrzebne do przetestowania funkcjonalności:
\begin{itemize}
 \item tworzymy profil studenta testowego,
 \item w zewnętrznym systemie zapisujemy studenta na kurs \textit{Algorytmy i struktury danych} i zaznaczamy jako przedmiot niezdany,
 \item w zakładce \textit{Tablica efektów} sprawdzamy, że przedmiot \textit{Algorytmy i struktury danych} jest w kolumnie przedmiotów niezrealizowanych,
 \item w zewnętrznym systemie zaznaczamy teraz, że kurs \textit{Algorytmy i struktury danych} został zaliczony,
 \item w zakładce \textit{Tablica efektów} sprawdzamy, że przedmiot \textit{Algorytmy i struktury danych} już jest w kolumnie przedmiotów zrealizowanych.
\end{itemize}

\section{Zgodność względem norm ISO/IEC 9126 i 25000}
TODO

\section{Plan beta testowania}
Przewidujemy dostarczyć projekt na początek semestru zimowego 2019/2020.
Aby dobrze przetestować \textit{Tablicę efektów}, udostępnimy program studentom drugiego i trzeciego roku oraz pracownikom dziekanatu na okres wakacyjny.
Dzięki temu w razie usterek będziemy mieć czas na naprawienie błędów, w tym czasie też studenci nie podejmują żadnych ostatecznych decyzji (np. związanych z głosowaniem, wyborem przedmiotów).

\section{Plan zarządzania ryzykiem}
Jak wspomniano w dokumencie \textit{Koncepcja wykonania systemu}, największe ryzyko dotyczące produktu to ryzyko opóźnień w harmonogramie.
Aby temu przeciwdziałać, postaramy się rozłożyć zadania programistyczne i testowe z naciskiem na pierwszą połowę czasu ustalonego w harmonogramie (patrz punkt 6.2.).
Ponieważ ostateczny termin wdrożenia produktu to połowa września (wtedy zdecydowana większość studentów zacznie interesować się zapisami na kolejny semestr), mamy dodatkowy miesiąc zapasu na ewentualne opóźnienia.

\section{Szczegółowe plany}
\subsection{Plan wykonania produktu}
Na początku nawiązany zostanie kontakt z osobami odpowiedzialnymi za rozwój Systemu Zapisów.
Po ich akceptacji, omówieniu ich roli w trakcie wykonywania projektu oraz wyznaczeniu programistów do realizacji rozpocznie się proces wytwarzania oprogramowania.
Będzie on zgodny ze standardem używanym w Systemie Zapisów (zawiera m. in. code review), uzupełnionym przez dodatkowego testera.

Po wytworzeniu produktu nastąpią dwie fazy testów.
W czasie fazy pierwszej sprawdzona zostanie zgodność produktu z Systemem Zapisów, jego współdziałanie z USOS, a także wydajność i inne wymagania niefunkcjonalne.
Po potwierdzeniu poprawności działania produktu przez ww. testy, rozpocznie się faza beta testowania (opisana w punkcie 4.).
W trakcie testowania błędy będą poprawiane zaraz po wykryciu.

Po zakończeniu testów osoby pracujące w dziekanacie zostaną przeszkolone do używania produktu.
W tym samym czasie produkt będzie wdrażany.

\subsection{Harmonogram i ocena pracochłonności}
Harmonogram jest następujący (wszystkie daty dotyczą roku 2019):
\begin{itemize}
 \item 1.02 -- 20.02 --- omówienie projektu z osobami odpowiedzialnymi za System Zapisów,
 \item 21.02 -- 31.05 --- wytwarzanie oprogramowania,
 \item 1.06 -- 30.06 --- testy, faza pierwsza,
 \item 1.07 -- 31.07 --- faza beta testowania,
 \item 1.08 -- 14.08 --- szkolenia użytkowników i wdrożenie produktu.
\end{itemize}
Niezależnie od powyższego harmonogramu, produkt zacznie być używany przez klientów w połowie września (wtedy studenci zaczynają się interesować ofertą zajęć na kolejny semestr).
Zatem czas od 15.08 do 14.09 jest dodatkowym buforem, który zostanie wykorzystany w razie opóźnień.

\section{Ocena zgodności wykonanych prac z wizją systemu i specyfikacją wymagań}
Zgodnie z głównym założeniem dotyczącym systemu dane wyświetlane studentom muszą być kompletne i przydatne.
Będzie to oceniane w trakcie beta testów, w trakcie których studenci oraz pracownicy dziekanatu będą zachęcani do wyrażenia swojej opinii na temat systemu.

Wizja wyglądu systemu nie jest formalnie ustalona.
Dopuszczamy wszelkie zmiany, o ile będą one komponować się wizualnie z istniejącymi już interfejsami Systemu Zapisów.


\end{document}
